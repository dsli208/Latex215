\documentclass{article}
\usepackage[utf8]{inputenc}
\usepackage{geometry}
\usepackage{amsmath}
\usepackage{physics}
\geometry{legalpaper, portrait, margin = 0.5in}
\rmfamily

\title{AMS 261 Checkpoint 1 Review Notes}
\author{David Li}
\date{September 2018}

\begin{document}

\maketitle

\section{Random Notes}

\par\noindent\large Parallel vectors, \textbf{projections}, unit vectors, (orthogonal vector formed by $\textbf{u} \times \textbf{v}$), inequalities, \textbf{partial derivatives}

\par\noindent\large Ask professor/TA about sketching \textbf{contour lines and level curves - REVIEW}
\par\noindent\Large Distance Formula: $D = \frac{PQ \cdot n}{\norm{n}}$ (point and plane use dot product, point and line will use cross product)
\par\noindent\large For limits on vectors, L'Hopitals rule is done \textbf{separately}
\par\noindent\large Study \#15 and \#16 of Hard WebAssign \#4, and \#2, \#8, and \#12 of Hard WebAssign \#5
\par\noindent\Large $a_{T} = \frac{a \cdot v}{\norm{v}}$ and $a_{N} = \sqrt{\norm{a}^{2} - a_{T}^{2}}$, $cos(\theta) = \frac{u \cdot v}{\norm{u}\norm{v}}$
\section{Selected Problems From Former Exams}

\subsection{Find a vector of magnitude 2 that points in the same direction as $-\textbf{i} + 3\textbf{j} + \textbf{k}$}
\par\noindent\large $\norm{v} = \sqrt{(-1)^{2} + (3)^{2} + (1)^{2}} = \sqrt{11}$, $2 = c\norm{v} = c\sqrt{11}$ so $c = \frac{2}{\sqrt{11}}$
\par\noindent\large The vector with $c$ times the magnitude is multiplied by $c$ so we get $-\frac{2}{\sqrt{11}}\textbf{i} + \frac{6}{\sqrt{11}}\textbf{j} + \frac{2}{\sqrt{11}}\textbf{k}$

\subsection{Determine the equation of the plane containing the points $P = (2, 0, -1)$, $Q = (0, 3, 1)$, and $R = (-1, 2, -1)$}

\par\noindent\large Assume vectors $PQ$ and $PR$ lie in the plane, $\textbf{u} = PQ = <-2, 3, 2>$ and $\textbf{v} = PR = <-3, 2, 0>$

\par\noindent\large To get the orthogonal normal vector, find the \textbf{cross product} between the two.
\begin{center}
$\textbf{n} = \textbf{u} \times \textbf{v} = \begin{vmatrix}
\textbf{i} & \textbf{j} & \textbf{k} \\ 
-2 & 3 & 2 \\ 
-3 & -2 & 0  \notag
\end{vmatrix} = -6\textbf{j} + 4\textbf{k} + 4\textbf{i} + 9\textbf{k} = 4\textbf{i} - 6\textbf{j} + 13\textbf{k} = <4, -6, 13>$ \end{center}
\par\noindent\large Now given the normal vector, plug into the standard equation $a(x - x_{0}) + b(y - y_{0}) + c(z - z_{0}) = 0$ using a point and the normal vector.  Assume our point is $P$ in this case.
\par\noindent\large $4(x - 2) - 6y + 13(z + 1) = 0$.  The general form is $4x - 8 - 6y + 13z + 13 = 4x - 6y + 13z + 5 = 0$

\subsection{Convert the equation $\rho = 2sec(\phi)$ to Cartesian coordinate and describe the surface}

\par\noindent\large This equation becomes $\rho = \frac{2}{cos(\phi)}$.  We can rearrange this to $\rho cos(\phi) = 2 = z$.
\par\noindent\large This means we get $z = 2$ in the end, so we have a plane at $z = 2$

\subsection{Let $r'(\theta) = 4cos(2\theta)\textbf{i} - 6sin(2\theta)\textbf{j}$ with $r(\pi) = \textbf{i} + 3\textbf{j}$.  Calculate $r(\theta)$}
\par\noindent\large Integrate $r'(\theta)$: $r(\theta) = \int 4cos(2\theta)\textbf{i} - 6sin(2\theta)\textbf{j} = 4\int cos(2\theta)\textbf{i} - 6\int sin(2\theta)\textbf{j} = \linebreak 4[\frac{sin(2\theta)}{2} + C_{i}]\textbf{i} - 6[-\frac{cos(2\theta)}{2} + C_{j}]\textbf{j} = (2sin(2\theta) + C_{i})\textbf{i} + (3cos(2\theta) + C_{j})\textbf{j}$ \vspace{0.25cm}

\par\noindent\large At $\theta = \pi$, $r(\pi) = (2sin(2\pi) + C_{i})\textbf{i} + (3cos(2\pi) + C_{j})\textbf{j} = C_{i}\textbf{i} + (3 + C_{j})\textbf{j}$
\par\noindent\large Setting the result $C_{i}\textbf{i} + (3 + C_{j})\textbf{j} = \textbf{i} + 3\textbf{j}$, we get $C_{i} = 1$ and $C_{j} = 0$
\par\noindent\large Therefore, $r(\theta) = (2sin(2\theta) +1)\textbf{i} + (3cos(2\theta))\textbf{j}$

\section{Textbook Review Problems}
\subsection{11.2.65 - Determine which of the vectors is/are parallel to $z$, if $z$ has initial point $(1, -1, 3)$ and terminal point $(-2, 3, 5)$.  (a) $-6\textbf{i} + 8\textbf{j} + 4\textbf{k}$ (b) $4\textbf{j} + 2\textbf{k}$}
\par\noindent\large By subtracting the two given points, we get the vector $<-3, 4, 2>$.  Our answer is (a) since the vector is a multiple of the above vector.

\subsection{When the projection of \textbf{u} onto \textbf{v} has the same magnitude as the project of \textbf{v} onto \textbf{u}, can you conclude that $\norm{u} = \norm{v}$?}

\par\noindent\Large Given $proj_{v}u = (\frac{u \cdot v}{\norm{v}^{2}})v$, $\norm{(\frac{u \cdot v}{\norm{v}^{2}})v} = \norm{(\frac{v \cdot u}{\norm{u}^{2}})u} \rightarrow \frac{\norm{u \cdot v}\norm{v}}{\norm{v}^{2}} = \frac{\norm{v \cdot u}\norm{u}}{\norm{u}^{2}}$
\par\noindent\large $\norm{u \cdot v} = \norm{v \cdot u}$ so those terms cancel leaving us, when simplified with $\norm{u} = \norm{v}$

\subsection{11.4.15 - Find a unit vector orthogonal to both $\textbf{u} = <4, -3, 1>$ and $\textbf{v} = <2, 5, 3>$}
\par\noindent\large To find an orthogonal vector, take the cross product of \textbf{u} and \textbf{v}:\vspace{0.25cm}
$\textbf{n} = \textbf{u} \times \textbf{v} = \begin{vmatrix}
\textbf{i} & \textbf{j} & \textbf{k} \\ 
4 & -3 & 1 \\ 
2 & 5 & 3  \notag
\end{vmatrix} = \linebreak -9\textbf{i} + 2\textbf{j} + 20\textbf{k} - 5\textbf{i} - 12\textbf{j} + 6\textbf{k}= -14\textbf{i} - 10\textbf{j} + 26\textbf{k} = <-14, -10, 26>$
\par\noindent\Large To get the unit vector, we take $\frac{u \times v}{\norm{u \times v}}$
\par\noindent\large $\norm{u \times v} = \sqrt{((-14)^{2} + (-10)^{2} + (26)^{2}} = 18\sqrt{3}$
\par\noindent\Large Our unit vector thus is $<-\frac{7}{9\sqrt{3}}, -\frac{5}{9\sqrt{3}}, \frac{13}{18\sqrt{3}}>$

\subsection{11.5.51 - Find an equation of the plane containing the lines given by $\frac{x - 1}{-2} = y - 4 = z$ and $\frac{x - 2}{-3} = \frac{y - 1}{4} = \frac{z - 2}{-1}$}
\par\noindent\Large Set $t = \frac{x - 1}{-2} = y - 4 = z$, solving we get parametric equations $x = -2t + 1$, $y = t + 4$, and $z = t$.  Similarly for the other line, $t = \frac{x - 2}{-3} = \frac{y - 1}{4} = \frac{z - 2}{-1}$.  Solving those, we get $x = -3t + 2$, $y = 4t + 1$, and $z = -t + 2$.\vspace{0.25cm}

\par\noindent\large We get from the two given lines vectors $\textbf{u} = <-2, 1, 1>$ and $\textbf{v} = <-3, 4, -1>$ respectively.  Similarly, we get points $P = (1, 4, 0)$ and $Q = (2, 1, 2)$ respectively.\vspace{0.25cm}

\par\noindent\large To get the normal vector of the plane, find $\textbf{u} \times \textbf{v} = \begin{vmatrix}
\textbf{i} & \textbf{j} & \textbf{k} \\ 
-2 & 1 & 1 \\ 
-3 & 4 & -1  \notag
\end{vmatrix} = -\textbf{i} - 3\textbf{j} - 8\textbf{k} - 4\textbf{i} - 2\textbf{j} + 3\textbf{k} = \linebreak -5\textbf{i} - 5\textbf{j} - 5\textbf{k} = <-5, -5, -5> or <1, 1, 1>$\vspace{0.25cm}

\par\noindent\large We can plug into our standard equation $a(x - x_{0}) + b(y - y_{0}) + c(z - z_{0}) = 0$ (assuming point $P$) to get $(x - 1) + (y - 4) + z = 0$.  This simplifies to $x + y + z - 5 = 0$

\subsection{11.5.111 - FInd a set of parametric equations for the line passing through the point $(0, 1, 4)$ that is perpendicular to $\textbf{u} = <2, -5, 1>$ and $\textbf{v} = <-3, 1, 4>$}

\par\noindent\large Find the orthogonal normal vector perpendicular to the plane: $\textbf{u} \times \textbf{v} = \begin{vmatrix}
\textbf{i} & \textbf{j} & \textbf{k} \\ 
2 & -5 & 1 \\ 
-3 & 1 & 4  \notag
\end{vmatrix} = \linebreak -20\textbf{i} - 3\textbf{j} + 2\textbf{k} - \textbf{i} - 8\textbf{j} - 15\textbf{k} = -21\textbf{i} - 11\textbf{j} - 13\textbf{k} = <-21, -11, -13>$ or $<21, 11, 13>$\vspace{0.25cm}

\par\noindent\large Given our vector and point passing through the line, we get the following parametric equations: $x = 21t$, $y = 1 + 11t$, and $z = 4 + 13t$

\subsection{10.4.25-34 - Convert the rectangular equation to polar form and sketch the graph: }
\begin{itemize}
    \item  $x^{2} + y^{2} = 9 \rightarrow r = 3$
    \item $x^{2} - y^{2} = 9 \rightarrow r^{2}cos^{2}(\theta) - r^{2}sin^{2}(\theta) = 9$ so $r = \frac{3}{\sqrt{cos^{2}(\theta) - sin^{2}(\theta)}}$
    \item $x^{2} + y^{2} = a^{2} \rightarrow r = a$
    \item $x^{2} + y^{2} -2ax = 0 \rightarrow r^{2} = 2arcos(\theta) \rightarrow r = 2acos(\theta)$
    \item $y = 8 \rightarrow rsin(\theta) = 8 \rightarrow r = 8csc(\theta)$
    \item $x = 12 \rightarrow rcos(\theta) = 12 \rightarrow r = 12sec(\theta)$
    \item $3x - y + 2 = 0 \rightarrow 3rcos(\theta) - rsin(\theta) + 2 = 0 \rightarrow 2 = rsin(\theta) - 3rcos(\theta) = r(sin(\theta) - 3cos(\theta)) = \linebreak \frac{2}{sin(\theta) - 3cos(\theta)}$
    \item $xy = 4 \rightarrow r^{2}sin(\theta)cos(\theta) = 4 \rightarrow r = 2\sqrt{sec(\theta)csc(\theta)}$
    \item $y^{2} = 9x \rightarrow r^{2}sin^{2}(\theta) = 9rcos(\theta) \rightarrow rsin^{2}(\theta) = 9cos(\theta) \rightarrow r = 9cot(\theta)csc(\theta)$
    \item $(x^{2} + y^{2})^{2} - 9(x^{2} - y^{2}) = 0 \rightarrow r^{4} - 9r^{2}(cos^{2}(\theta) - sin^{2}(\theta)) = 0 \rightarrow r^{4} = 9r^{2}(cos^{2}(\theta) - sin^{2}(\theta)) = \linebreak r^{2} = 9(cos^{2}(\theta) - sin^{2}(\theta)) \rightarrow r = 3\sqrt{(cos^{2}(\theta) - sin^{2}(\theta))}$
\end{itemize}

\subsection{10.4.35-44 - Convert the polar equation to rectangular form and sketch the graph}
\begin{itemize}
    \item $r = 4 \rightarrow x^{2} + y^{2} = 16$
    \item $r = -1 \rightarrow x^{2} + y^{2} = 1$
    \item $r = 3sin(\theta) \rightarrow r^{2} = 3rsin(\theta) = 3y$
    \item $r = 5cos(\theta) \rightarrow r^{2} 5rcos(\theta) = 5x$
    \item $r = \theta \rightarrow tan^{-1}(r) = tan^{-1}(\frac{y}{x}) = tan^{-1}(x^{2} + y^{2}) \leftarrow$ \textbf{REVIEW}
    \item $\theta = \frac{5\pi}{6} \rightarrow \leftarrow$ \textbf{REVIEW}
    \item $r = 3sec(\theta) = \frac{3}{cos(\theta)} \rightarrow rcos(\theta) = x = 3$
    \item $r = -6csc(\theta) = -\frac{6}{sin(\theta)} = rsin(\theta) = y = -6$
    \item $r = sec(\theta)tan(\theta) = \frac{1}{cos(\theta)}\frac{sin(\theta)}{cos(\theta)} \rightarrow rcos^{2}(\theta) = sin(\theta) \rightarrow r^{2}cos^{2}(\theta) = (rcos(\theta))^{2} = rsin(\theta) \rightarrow x^{2} = y$ 
    \item $r = cot(\theta)csc(\theta) = \frac{cos(\theta)}{sin(\theta)}\frac{1}{sin(\theta)} \rightarrow rsin^{2}(\theta) = cos(\theta) \rightarrow r^{2}sin^{2}(\theta) = (rsin(\theta))^{2} = rcos(\theta) \rightarrow y^{2} = x$
\end{itemize}

\subsection{11.7.71-76 - Match the equation, written in terms of cylindrical or spherical coordinates with its graph (in textbook)}
\begin{itemize}
    \item $r = 5 \rightarrow$ d
    \item $\theta = \frac{\pi}{4} \rightarrow$ e
    \item $\rho = 5 \rightarrow$ c
    \item $\phi = \frac{\pi}{4} \rightarrow$ a
    \item $r^{2} = z \rightarrow$ f
    \item $\rho = 4sec(\phi) \rightarrow$ b
\end{itemize}

\subsection{11.7.99 - Find inequalities that describe the solid and state the coordinate system used.  Position the solid on the coordinate system such that the inequalities are as simple as possible.  For the solid inside both $x^{2} + y^{2} + z^{2} = 9$ and $(x - \frac{3}{2})^{2} + y^{2} = \frac{9}{4}$}
\par\noindent\large Using \textbf{cylindrical coordinates}, we get the following inequalities:
\par\noindent\Large $-\frac{\pi}{2} \leq \theta \leq \frac{\pi}{2}, -\sqrt{9 - r^{2}} \leq z \leq \sqrt{9 - r^{2}}, 0 \leq r \leq 3cos(\theta)$\vspace{0.25cm}
\par\noindent\large
\begin{itemize}
    \item For $\theta$, the cylinder intersects the \textbf{right} half of the sphere, in quadrants I and IV
    \item To get $r$ (its max value), expand the $(x - \frac{3}{2})^{2}$ part of the second equation.  We get $x^{2} - 3x + y^{2} = 0$ (we will get $\frac{9}{4}$ on both sides that cancel themselves out), and this rearranges to $x^{2} + y^{2} = 3x$.  Converting to cylindrical/polar form, we get $r^{2} = 3rcos(\theta)$, which simplifies to $r = 3cos(\theta)$ (note that since the cylinder is from the $z$ axis, the lower bound will simply be for $r$, the center line of the sphere).
    \item We can carry our result from finding the value of $r$ into finding our bounds for $z$: because $x^{2} + y^{2} = r^{2}$, we can plug that into the first equation to get $r^{2} + z^{2} = 9$.  By simply solving for $z$, we get $\pm\sqrt{9 - r^{2}}$ as our bounds for $z$
\end{itemize}

\subsection{12.1, Example 4 - Sketch the space curve C represented by the intersection of the semiellipsoid $\frac{x^{2}}{12} + \frac{y^{2}}{24} + \frac{z^{2}}{4} = 1, z \geq 0$ and the parabolic cylinder $y = x^{2}$.  Then find a vector-valued function to represent the graph.}
\par\noindent\Large Using parameter $x = t$, and plugging this into $y = x^{2}$, we get $y = t^{2}$.  Our original equation becomes $\frac{t^{2}}{12} + \frac{t^{4}}{24} + \frac{z^{2}}{4} = 1, z \geq 0$.  To get $z$, rearrange the equation to get $\frac{z^{2}}{4} = 1 - \frac{t^{2}}{12} - \frac{t^{4}}{24}$, which rearranges to $z^{2} = 4 - \frac{t^{2}}{3} - \frac{t^{4}}{6} = \frac{24 - 2t^{2} - t^{4}}{6} = \frac{(6 + t^{2})(4 - t^{2})}{6}\linebreak$  Therefore, $z = \sqrt{\frac{(6 + t^{2})(4 - t^{2})}{6}}$, and because we have the precondition that $z \geq 0$, we set $\sqrt{\frac{(6 + t^{2}(4 - t^{2})}{6}} \geq 0$, which simply becomes $\frac{(6 + t^{2})(4 - t^{2})}{6} \geq 0$.  Since $6 + t^{2}$ can only be positive, we focus on when $4 - t^{2} \geq 0$.  Rearranging this inequality becomes $4 \geq t^{2}$, which in turn becomes $-2 \leq t \leq 2$.  Our final vector valued equation is thus $r(t) = t\textbf{i} + t^{2}\textbf{j} + \sqrt{\frac{(6 + t^{2}(4 - t^{2})}{6}}\textbf{k}$

\subsection{Find $r(t)$ that satifies the initial conditions: $r''(t) = -32\textbf{j}$, $r'(0) = 600\sqrt{3}\textbf{i} + 600\textbf{j}$, $r(0) = 0$}
\par\noindent\large $r'(t) = \int -32\textbf{j} = c_{i}\textbf{i} -32\int dt\textbf{j} = c_{i}\textbf{i} - [32t + c_{j}]\textbf{j} = c_{i}\textbf{i} + [-32t + c_{j}]\textbf{j}$.  Plugging in $t = 0$, we get $r'(0) = c_{i}\textbf{i} + [-32(0) + c_{j}]\textbf{j} = c_{i}\textbf{i} + c_{j}\textbf{j} = 600\sqrt{3}\textbf{i} + 600\textbf{j}$.  We can easily match $c_{i} = 600\sqrt{3}$ and $c_{j} = 600$, so we get $r'(t) = 600\sqrt{3}\textbf{i} + [-32t + 600]\textbf{j}$.\vspace{0.25cm}

\par\noindent\large $r(t) = \int 600\sqrt{3}\textbf{i} + [-32t + 600]\textbf{j} dt = \int 600\sqrt{3}dt\textbf{i} + \int [-32t + 600]dt\textbf{j} = (600\sqrt{3}t + C_{i})\textbf{i} + [-16t^{2} + 600t + C_{j}]\textbf{j}\linebreak$
$r(0) = C_{i}\textbf{i} + C_{j}\textbf{j} = 0$, so $C_{i} = C_{j} = 0$ and $r(t) = 600\sqrt{3}t\textbf{i} + [-16t^{2} + 600{t}]\textbf{j}$

\subsection{12.4.43 - An object moves along the path given by $r(t) = 3t\textbf{i} + 4t\textbf{j}$.  Find $v(t), a(t), T(t)$, and $N(t)$ (if it exists).  What is the form of the path.  Is the speed of the object constant or changing?}
\par\noindent\Large $v(t) = r'(t) = 3\textbf{i} + 4\textbf{j}$, $a(t) = r''(t) = 0$, $T(t) = \frac{r'(t)}{\norm{r'(t)}} = \frac{<3, 4>}{\sqrt{(3)^{2} + (4)^{2}}} = \frac{3}{5}\textbf{i} + \frac{4}{5}\textbf{j}$, $\linebreak N(t) = \frac{T'(t)}{\norm{T'(t)}} = \frac{0}{\sqrt{(\frac{3}{5})^{2} + (\frac{4}{5})^{2}}} = \frac{0}{\sqrt{\frac{9}{25} + \frac{16}{25}}} = \frac{0}{\sqrt{\frac{25}{25}}} = \frac{0}{\sqrt{1}} = 0$\vspace{0.25cm}

\par\noindent\large Given $a(t) = 0$, there is no change in velocity, and thus no change in speed: \textbf{speed is CONSTANT}
\subsection{13.1.57 - Describe the level curves of the function and sketch a contour map of the surface using level curves for the given values of $c$: $f(x, y) = \frac{x}{x^{2} + y^{2}}$, $c = \pm\frac{1}{2}, \pm 1, \frac{3}{2}, \pm 2$}
\par\noindent\large To begin, by definition a level curve is a curve at a level $c$ containing all points on that level.
\par\noindent By that, let's set $f(x, y) = c = \frac{x}{x^{2} + y^{2}}$.  Rearranging the equation, we can get $c(x^{2} + y^{2}) = x$, which becomes $cx^{2} - x + cy^{2} = 0$.  By dividing all terms by $c$, this becomes $x^{2} - \frac{x}{c} + y^{2} = 0$.  Completing the square on the $x$ terms makes this $(x - \frac{1}{2c})^{2} + y^{2}  - (\frac{1}{2c})^{2} = 0$, which in turn becomes $(x - \frac{1}{2c})^{2} + y^{2} = (\frac{1}{2c})^{2}$.  This is our final equation; it resembles a \textbf{circle} so for each value of $c$, plot the graph accordingly.

\section{Recitation Problems}

\subsection{13.1.4 - Explain how to sketch a contour map of a function $f(x, y)$}
\par\noindent\large Given a value $c$ that $f(x, y)$ is equal to, assuming level curves, set $f(x, y) = c$ and manipulate the equation; you should be able to get it into a form such that it will represent a familiar shape

\subsection{13.1.64 - All of the level curves of the surface given by $z = f(x, y)$ are concentric circles.  Does this imply that the graph of $f$ is a hemisphere?  Illustrate our answer with an example.}

\par\noindent\large No, because it could be on one plane, \textbf{or the same circle}.

\subsection{13.1.87 - Meteorologists measure the atmospheric pressure in millibars.  From these observations, they create weather maps on which the curves of equal atmospheric pressure (isobars) are drawn.  On the map, the closer the isobars, the higher the wind speed.  Match points A, B, and C with (a) highest pressure, (b) lowest pressure, and (c) highest wind velocity.}

\par\noindent\large A - lowest pressure, B - highest velocity, C - highest pressure

\subsection{13.2.77 - Use spherical coordinates to find the limit: $\lim_{(x,y)\rightarrow (0, 0, 0)} \frac{xyz}{x^{2} + y^{2} + z^{2}}$}

\par\noindent\Large $\lim_{(x,y)\rightarrow (0, 0, 0)} \frac{xyz}{x^{2} + y^{2} + z^{2}} = \lim_{(x,y)\rightarrow (0, 0, 0)} \frac{(\rho sin(\phi)cos(\theta))(\rho sin(\phi)sin(\theta))(\rho cos(\phi))}{(\rho sin(\phi)cos(\theta)^{2} + (\rho sin(\phi)sin(\theta))^{2} + (\rho cos(\phi))^{2}} = \linebreak \lim_{(x,y)\rightarrow (0, 0, 0)} \frac{\rho^{3}sin^{2}(\phi)cos(\phi)sin(\theta)cos(\theta)}{\rho^{2}sin^{2}(\phi)cos^{2}(\theta) + \rho^{2}sin^{2}(\phi)sin^{2}(\theta) + \rho^{2}cos^{2}(\phi)} = \lim_{(x,y)\rightarrow (0, 0, 0)} \frac{\rho^{3}sin^{2}(\phi)cos(\phi)sin(\theta)cos(\theta)}{\rho^{2}sin^{2}(\phi) + \rho^{2}cos^{2}(\phi)} = \linebreak\lim_{(x,y)\rightarrow (0, 0, 0)} \rho sin^{2}(\phi)cos(\phi)sin(\theta)cos(\theta) = 0$

\end{document}
