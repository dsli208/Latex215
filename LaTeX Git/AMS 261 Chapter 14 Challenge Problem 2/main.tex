\documentclass{article}
\usepackage[utf8]{inputenc}
\usepackage{geometry}
\usepackage{amsmath}
\usepackage{physics}
\usepackage{graphicx}
\usepackage{textcomp}
\usepackage{hyperref}
\geometry{legalpaper, portrait, margin = 0.5in}
\rmfamily

\title{AMS 261 Chapter 14 Challenge Problem 2}
\author{David S. Li (SBUID: 110328771)}
\date{November 15, 2018}

\begin{document}

\maketitle

\section{A company manufactures cylindrical rods (right, circular cylinders) with radius $R$ cm, height $H$ cm, and mass density function (in g cm\textsuperscript{-3}) $1 - \frac{r^{2}}{R^{2}}$, where $r$ is the distance from the rod's axis (in cm).  A customer wants rods with mass 50 g and minimal surface area.  Determine the dimensions $R$ and $H$ of that rod.}

\par\noindent\Large Basically, we are trying to minimize surface area, with $m = 50$ g as our constraint.\vspace{0.25cm}

\par\noindent\Large We can set our mass density function equal to $\rho = 1 - \frac{r^{2}}{R^{2}}$ and $m = 50 = \iiint_{R}\rho dA$\vspace{0.25cm}

\par\noindent\Large To evaluate what we can so far of the integral, let's describe the region of integration $R$: We can easily tell that $0 \leq r \leq R$, $0 \leq z \leq H$, and $0 \leq \theta \leq 2\pi$.\vspace{0.25cm}

\par\noindent\Large Integrating, we get $\iiint_{R}\rho dA = \int_{0}^{H}\int_{0}^{2\pi}\int_{0}^{R}(1 - \frac{r^{2}}{R^{2}})rdrd\theta dz = \int_{0}^{H}\int_{0}^{2\pi}\int_{0}^{R}(r - \frac{r^{3}}{R^{2}})drd\theta dz = \int_{0}^{H}\int_{0}^{2\pi}[\frac{r^{2}}{2} - \frac{r^{4}}{4R^{2}}]_{0}^{R}d\theta dz = \int_{0}^{H}\int_{0}^{2\pi}\frac{R^{2}}{2} - \frac{R^{4}}{4R^{2}}d\theta dz = \int_{0}^{H}\int_{0}^{2\pi}\frac{R^{2}}{2} - \frac{R^{2}}{4}d\theta dz = \int_{0}^{H}\int_{0}^{2\pi}\frac{R^{2}}{4}d\theta dz = \int_{0}^{H}\frac{R^{2}}{4}[\theta]_{0}^{2\pi}dz = \int_{0}^{H}\frac{R^{2}\pi}{2}dz = \frac{R^{2}\pi}{2}[z]_{0}^{H} = \frac{\pi R^{2}H}{2}$\vspace{0.25cm}

\par\noindent\Large Surface area of a right circular cylinder is $A_{S} = F(R, H) = 2\pi RH + 2\pi R^{2}$.  Our constraint equation is $G(R, H) = 50 = \frac{\pi R^{2}H}{2}$ or $G(R, H) = \pi R^{2}H - 100$ after rearranging.\vspace{0.25cm}

\par\noindent\Large We can do Lagrange Multipliers to solve for our variables: $\nabla F(R, H) = \lambda\nabla G(R, H)$
\par\noindent\Large $\nabla F(R, H) = \frac{\partial F}{\partial R}\textbf{i} + \frac{\partial F}{\partial H}\textbf{j}$, and similarly $\nabla G(R, H) = \frac{\partial G}{\partial R}\textbf{i} + \frac{\partial G}{\partial H}\textbf{j}$.
\par\noindent\Large We get $\nabla F(R, H) = (2\pi H + 4\pi R)\textbf{i} + 2\pi R\textbf{j}$ and $\nabla G(R, H) = 2\pi RH\textbf{i} + \pi R^{2}\textbf{j}$.\vspace{0.25cm}

\par\noindent\Large Therefore, we get the following system of equations:
\begin{itemize}
    \item $2\pi H + 4\pi R = 2\pi RH\lambda$ (1, \textbf{i} equation)
    \item $2\pi R = \lambda\pi R^{2}$ (2, \textbf{j} equation)
    \item $\pi R^{2}H = 100$ (3, constraint)
\end{itemize}

\par\noindent\Large To start, solve for $\lambda$ in equation (2): Dividing both sides by $\pi R^{2}$ will give us $\lambda = \frac{2}{R}$
\par\noindent\Large Now, factor $2\pi$ out of equation (1): $2\pi(H + 2R) = 2\pi(RH\lambda)$, so $H + 2R = RH\lambda$.  Substituting $\frac{2}{R}$ for $\lambda$ gives us $H + 2R = 2H$, which simplifies to $2R = H$.  We can substitute this into (3) to give us $2\pi R^{3} = 100$, and rearranging this gives us $\pi R^{3} = 50 \rightarrow R^{3} = \frac{50}{\pi} \rightarrow R = \sqrt[3]{\frac{50}{\pi}} \approx 2.515$ cm.  To get $H$, simply use the equation $2R = H$, and plug in for $R$; we get $H \approx 5.031$ cm.  (Note that you can also plug into (1) or (3) once given the value of $R$ to get your value of $H$)\vspace{0.25cm}

\par\noindent\Large Therefore, our ideal values of $R$ and $H$ that minimize surface area as $R = \sqrt[3]{\frac{50}{\pi}} \approx 2.515$ cm and $H = 2\sqrt[3]{\frac{50}{\pi}} \approx 5.031$ cm.

\end{document}
