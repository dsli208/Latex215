\documentclass{article}
\usepackage[utf8]{inputenc}
\usepackage{geometry}
\usepackage{amsmath}
\usepackage{physics}
\usepackage{graphicx}
\usepackage{textcomp}
\usepackage{hyperref}
\geometry{legalpaper, portrait, margin = 0.5in}
\rmfamily

\title{AMS 261 HW11 (Hard)}
\author{David S. Li (SBUID: 110328771)}
\date{November 30, 2018}

\begin{document}

\maketitle

\section{15.2.2 - Describe how reversing the orientation of a curve $C$ affects $\int_{C}\textbf{F}\cdot d\textbf{r}$}

\par\noindent\Large Just with reversing the interval of a one-dimensional integral, reversing the orientation of a curve will produce the \textbf{negative} of the associated line integral (i.e., reversing the orientation of $C$ will yield $-\int_{C}\textbf{F}\cdot d\textbf{r}$).

\section{15.3.2 (Independence of Path) - What does it mean for a line integral to be independent of path?  State the method for determining whether a line integral is independent of path.}

\par\noindent\Large As per Theorem 15.6 in the textbook, a line integral is independent of path if the following two conditions are satisfied: (i) The vector $\textbf{F}$ must be \textbf{continuous on an open connected region}, and (ii) $\textbf{F}$ must also be \textbf{conservative}.\vspace{0.25cm}

\par\noindent\Large If the first condition holds, to find that $\textbf{F}$ is conservative, we are given the following:
\par\noindent\Large $f(x, y) = \int_{C}\textbf{F}\cdot d\textbf{r} = \int_{C}Mdx + Ndy$.  Remember that $\textbf{F}$ is conservative when $\frac{\partial M}{\partial y} = \frac{\partial N}{\partial x}$, or alternatively, $f_{x} = f_{y}$.

\section{15.4.44 - The figure shows a region $R$ bounded by a piecewise smooth simple closed path $C$.}
\subsection{(a) - Is $R$ simply connected?  Explain.}

\par\noindent\Large Yes, because \textbf{$R$ is a single closed region with no holes}.

\subsection{(b) - Explain why $\int_{C}f(x)dx + g(y)dy = 0$ where $f$ and $g$ are differentiable functions.}

\par\noindent\Large Assuming we are talking about the same region as stated in the original problem, remember that $C$ represents a \textbf{closed} path, where $\int_{C}\textbf{F}\cdot d\textbf{r} = 0$, as per Theorem 15.7 in the textbook.

\section{15.5.44 - Determine how the graph of the surface $s(u, v) = ucos(v)\textbf{i} + u^{2}\textbf{j} + usin(v)\textbf{k}$ differs from the graph of $r(u, v) = ucos(v)\textbf{i} + usin(v)\textbf{j} + u^{2}\textbf{k}$, where $0 \leq u \leq 2$ and $0 \leq v \leq 2\pi$, as shown in the figure.  (It is not necessary to graph $s$.)}

\par\noindent\Large The "elliptical cross section" of the shape will now be facing in the $+y$-direction, as opposed to the $+z$-direction in the original graph.  In other words, the graph, when going outwards, will go on and revolve around the $+y$-axis, as opposed to the $+z$-axis.

\end{document}
