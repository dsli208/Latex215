\documentclass{article}
\usepackage[utf8]{inputenc}
\usepackage{geometry}
\usepackage{amsmath}
\geometry{legalpaper, portrait, margin = 0.5in}
\rmfamily

\title{AMS 261 HW3 (Hard)}
\author{David S. Li (SBUID: 110328771)}
\date{September 14, 2018}

\begin{document}

\maketitle

\section{11.6.2 - What does the equation $z = x^{2}$ represent in the $xz$-plane?  What does it represent in three-space?}

\par\noindent\Large $z = x^{2}$ represents a parabola in the $+z$ plane, similar to how $y = x^{2}$ would look in the $xy$ plane.  In 3D space, it would represent a parabolic surface with no $y$-dimension; the aforementioned equation would be the generating curve for that surface.
\section{11.6.14 - Describe and sketch the surface: $y^{2} - z^{2} = 25$}

\par\noindent\Large The equation will become a hyperbola on the $yz$ plane.  Based on the standard equation for a hyperbola: $c^{2} = a^{2} + b^{2}$, the equation could be rewritten as $(5)^{2} = (y - 0)^{2} + (z - 0)^{2}$, where $c = 5$ represents half the distance between the two foci.  Rearranging, we would get the equation $z^{2} = \pm\sqrt{y^{2} - 25}$.  \textbf{See the FIRST drawing on the back for the sketch.}

\section{11.7.77 - Explain why in spherical coordinates the graph of $\theta = c$ is a half plane and not an entire plane.}
\par\noindent\Large The graph of $\theta = c$ is a half plane because $\theta$ is relative to the center.  From there, you will usually get the same view given the sphere is uniform, and with each view, you would only be able to see half the plane ahead of you.

\section{11.7.92 - Sketch the solid that has the given description in spherical coordinates: $0 \leq \theta \leq 2\pi$, $\frac{\pi}{4} \leq \phi \leq \frac{\pi}{2}$, $0 \leq \rho \leq 1$}
\par\noindent\Large \textbf{See the SECOND drawing on the back for the sketch.}

\section{11.7.98 - The solid that remains after a hole 1 in. in diameter is drilled through the center of a sphere 6 in. in diameter.  Find inequalities that describe the solid and state the coordinate system used.  Position the solid on the coordinate system such that the inequalities are as simple as possible.}

\par\noindent\Large Spherical coordinate system, $0 \leq \theta \leq 2\pi$, $0 \leq \phi \leq \frac{\pi}{4}$, $0.5 \leq \rho \leq 3$



\end{document}
