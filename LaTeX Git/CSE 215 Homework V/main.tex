\documentclass{article}
\usepackage[utf8]{inputenc}

\usepackage{geometry}
\geometry{total = {150mm, 260mm}, left = 20mm, top = 15mm, right = 20mm, bottom = 20mm}
\title{CSE 215 Homework V}
\author{David S. Li (SBU ID 110328771)}
\date{December 12, 2016}

\begin{document}

\maketitle

\section{$F_{0}, F_{1}, ...$ represents the Fibonacci sequence, prove $F_{k + 1}^{2} - F_{k}^{2} - F_{k - 1}^{2} = 2F_{k}F_{k - 1}$ $\forall k \geq 1$}

We are dealing with a Fibonacci sequence.  From that we know that $F_{0} = 1$ and $F_{1} = 1$.  Additionally, we know that $F_{n} = F_{n - 1} + F_{n - 2}$ $\forall n \geq 2$ where $n \in Z$ \par\vspace{0.5cm}

\noindent We want to prove (by induction) $F_{k + 1}^{2} - F_{k}^{2} - F_{k - 1}^{2} = 2F_{k}F_{k - 1}$ $\forall k \geq 1$ \par\vspace{0.5cm}

\noindent Base case (k = 1):  Show $F_{1 + 1}^{2} - F_{1}^{2} - F_{1 - 1}^{2} = 2F_{1}F_{1 - 1}$ or simplified, $F_{2}^{2} - F_{1}^{2} - F_{0}^{2} = 2F_{1}F_{0}$
\par\noindent
$F_{2} = F_{1} + F_{0} = 1 + 1 = 2$ and substituting into the above equation, $2^{2} - 1^{2} - 1^{2} = 2 * 1 * 1$.  We get $2 = 2$ on both sides, thereby proving the base case true. \par\vspace{0.5cm}

\noindent From our base case, we suppose that the property P(k) is the equation $F_{k + 1}^{2} - F_{k}^{2} - F_{k - 1}^{2} = 2F_{k}F_{k - 1}$ $\forall k \geq 1$.  So for our inductive step, we want to show that P(k + 1) is also true.  That means \par\noindent $F_{(k + 1) + 1}^{2} - F_{k + 1}^{2} - F_{(k + 1) - 1}^{2} = 2F_{k + 1}F_{(k + 1) - 1}$ $\forall k \geq 1$, or $F_{k + 2}^{2} - F_{k + 1}^{2} - F_{k}^{2} = 2F_{k + 1}F_{k}$ $\forall k \geq 1$ \par\vspace{0.5cm}

\noindent Substituting our original Fibonacci equation for each of the terms on the left side of the above equation, we get: \par\noindent
$F_{k + 2} = F_{(k + 2) - 1} + F_{(k + 2) - 2} = F_{k + 1} + F_{k}$, $F_{k + 1} = F_{k} + F_{k - 1}$, and $F_{k} = F_{k - 1} + F_{k - 2}$ $\forall k \geq 1$ \par\vspace{0.5cm}

\noindent Now, squaring these terms, we get: $F_{k + 2}^{2} = (F_{k + 1} + F_{k})^{2} = F_{k + 1}^{2} + 2F_{k + 1}F_{k} + F_{k}^2$,  \par\noindent
$F_{k + 1}^{2} = (F_{k} + F_{k - 1})^{2} = F_{k}^{2} + 2F_{k}F_{k - 1} + F_{k - 1}^{2}$, $F_{k}^{2} = (F_{k - 1} + F_{k - 2})^{2} = F_{k - 1}^{2} + 2F_{k - 1}F_{k - 2} + F_{k - 2}^{2}$ \par\vspace{0.5cm}

\noindent Substituting those values into the left hand side of the equation for P(k + 1), we get \par\noindent
$F_{k + 2}^{2} - F_{k + 1}^{2} - F_{k}^{2} = F_{k + 1}^{2} + 2F_{k + 1}F_{k} + F_{k}^2 - (F_{k}^{2} + 2F_{k}F_{k - 1} + F_{k - 1}^{2}) - (F_{k - 1}^{2} + 2F_{k - 1}F_{k - 2} + F_{k - 2}^{2})$ \par\noindent
$= F_{k + 1}^{2} + 2F_{k + 1}F_{k} + F_{k}^{2} - F_{k}^{2} - 2F_{k}F_{k - 1} - F_{k - 1}^{2} - F_{k - 1}^{2} - 2F_{k - 1}F_{k - 2} - F_{k - 2}^{2} = $\par\noindent
$F_{k}^{2} + 2F_{k}F_{k - 1} + F_{k - 1}^{2} + 2F_{k + 1}F_{k} - 2F_{k}F_{k - 1} - F_{k - 1}^{2} - (F_{k - 1}^{2} + 2F_{k - 1}F_{k - 2} + F_{k - 2}^{2}) = $ \par\noindent
$F_{k}^{2} + 2F_{k}F_{k - 1} + F_{k - 1}^{2} + 2F_{k + 1}F_{k} - 2F_{k}F_{k - 1} - F_{k - 1}^{2} - F_{k}^{2} = 2F_{k + 1}F_{k}$
\par\vspace{0.5cm}
\noindent Therefore, we have proven $F_{k + 2}^{2} - F_{k + 1}^{2} - F_{k}^{2} = 2F_{k + 1}F_{k}$, thereby proving P(k + 1) and thus proving our original statement, that $F_{k + 1}^{2} - F_{k}^{2} - F_{k - 1}^{2} = 2F_{k}F_{k - 1}$ $\forall k \geq 1$

\section{Given $w_{1} = 1$, $w_{2} = 2$, and $w_{k} = w_{k - 2} + k$ $\forall k \geq 2$, prove $w_{k} = w_{k - 2} + k$ $\forall k \geq 2$}

Given $w_{1} = 1$ and $w_{2} = 2$, let's test $w_{k} = w_{k - 2} + k$ for a few values of k (k = 3): \par\noindent
$w_{3} = w_{3 - 2} + 3 = w_{1} + 3 = 2 + 2 = 4$, $w_{4} = w_{4 - 2} + 4 = w_{2} + 4 = 2 + 2 + 2 = 6$, \par\noindent $w_{5} = w_{5 - 2} + 5 = w_{3} + 5 = 2 + 2 + 2 + 3 = 9$, $w_{6} = w_{6 - 2} + 6 = w_{4} + 6 = 6 + 6 = 2 + 2 + 2 + 3 + 3 = 12$, $w_{7} = w_{7 - 2} + 7 = w_{5} + 7 = 9 + 7 = 2 + 2 + 2 + 3 + 3 + 4 = 16$ \par\vspace{0.5cm}

\noindent If we keep this pattern, we observe the following \par\noindent
For EVEN numbers: $w_{k} = 2 + 2 * \sum_{2}^{\frac{k}{2}}n$, for ODD numbers: $w_{k} = 2 + 2 * \sum_{2}^{\frac{k - 1}{2}}n + (\frac{k + 1}{2} + 1)$ \par\vspace{0.5cm}

\noindent BASE CASES (One even and one odd): \par\noindent
(n = 3, n is odd): $w_{3} = 2 + 2 * \sum_{2}^{\frac{3 - 1}{2}}n + \frac{3 + 1}{2}= 2 + 2 * \sum_{2}^{2}n = 2 + 2 * 0 + \frac{3 + 1}{2} = 2 + 0 + 2 = 4$ \par\noindent
(n = 4, n is even) $w_{4} = 2 + 2 * \sum_{2}^{\frac{4}{2}}n = 2 + 2 * (2) = 2 + 4 = 6$ \par\vspace{0.5cm}

\noindent So suppose $\forall$ $3 \leq i \leq k$, $P(i) = w_{i} = 2 + 2 * \sum_{2}^{\frac{i}{2}}n$ if $i$ is even, or $P(i) = w_{i} = 2 + 2 * \sum_{2}^{\frac{i - 1}{2}}n + (\frac{i + 1}{2} + 1)$ if $i$ is odd \par\vspace{0.5cm}
\noindent For the inductive step, prove $P(k + 1)$ based on the following cases: \par\vspace{0.5cm}

\noindent Case (1): $k$ is EVEN, therefore $k + 1$ is ODD \par\noindent
Show $w_{k + 1} = 2 + 2 * \sum_{2}^{\frac{k}{2}}n + (\frac{k}{2} + 2)$
\par\noindent
By definition of even, $k = 2a$ where $a \in Z$, and by definition of odd $k + 1 = 2a + 1$  
\par\noindent
$w_{k + 1} = 2 + 2 * \sum_{2}^{\frac{(2a + 1) - 1}{2}}n + (\frac{(2a + 1) + 1}{2} + 1) =  2 + 2 * \sum_{2}^{\frac{(2a + 1) - 1}{2}}n + (\frac{(2a + 1) + 1}{2} + 1) = 2 + 2 * \sum_{2}^{\frac{2a}{2}}n + (\frac{2a + 2}{2} + 1)$ \par\noindent
$ = 2 + 2 * \sum_{2}^{a}n + (a + 2)$\par\noindent
Remember that $k = 2a$, we can say by dividing both sides by 2 that $a = \frac{k}{2}$ \par\noindent Substituting $a$ for $\frac{k}{2}$, we get $2 + 2 * \sum_{2}^{\frac{a}{2}}n + (\frac{k}{2} + 2) = w_{k + 1}$.  Therefore, $P(k + 1)$ is true for case (1), when $k$ is even. \par\vspace{0.5cm}

\noindent Case (2): $k$ is ODD, therefore $k + 1$ is EVEN \par
\noindent Show $w_{k + 1} = 2 + 2 * \sum_{2}^{\frac{k + 1}{2}}n$ \par
\noindent By definition of odd and even respectively, $k = 2a + 1$ and $k + 1 = 2a + 2$, where $a \in Z$ \par\noindent
$w_{k + 1} = 2 + 2 * \sum_{2}^{\frac{2a + 2}{2}}n = 2 + 2 * \sum_{2}^{a + 1}n$ \par\noindent
By rearranging the equation $k = 2a + 1$, we get $a = \frac{k - 1}{2}$.  Substituting that value back in for $a$ we get: \par\noindent
$2 + 2 * \sum_{2}^{\frac{k - 1}{2} + \frac{2}{2}}n = 2 + 2 * \sum_{2}^{\frac{k + 1}{2}} = w_{k + 1}$.  Therefore, $P(k + 1)$ is true for case (2), when $k$ is odd \par\vspace{0.5cm}

\noindent Because $P(k + 1)$ is true for both odd and even numbers, we have proven $P(k + 1)$ true, and therefore we have proven that $w_{k} = w_{k - 2} + k$ $\forall k \geq 2$ is a true statement.

\section{Set recursion}

\subsection{3a. Given the recursive set definition, $S \subseteq N$, where $1 \in S$, and $x \in S \rightarrow x + 2\sqrt{x} + 1 \in S$, rewrite in set-builder notation}

$S$ $= \{x + 2\sqrt{x} + 1 \mid \sqrt{x} \in N\}$
\par\vspace{0.5cm}

\subsection{3b. $S = \{2^{n} - 2 \mid n \in N\}$.  Rewrite recursively}

$n \in N \rightarrow 2^{n} - 2 \in S$

\section{Solve the LHSOR: Where $n \in N$, $f(n) = 0$ if $n = 0$, $1$ if $n = 1$, and $f(n - 1) + f(n - 2)$ otherwise}

When $n \neq 0$ and $n \neq 1$, $f(n) = f(n - 1) + f(n - 2)$ and $A = 1$, $B = 1$. \par
\noindent The recurrence relation is $x^{2} = x + 1$.  We can rearrange this to get $x^{2} - x - 1$. \par\vspace{0.5cm}

\noindent To get the roots of this equation, we need to use the quadratic formula: $x = \frac{1 \pm \sqrt{(-1)^{2} - 4(1)(-1)}}{2(1)}$ \par\noindent
x = $\frac{1 \pm \sqrt{1 + 4}}{2} = \frac{1 \pm \sqrt{5}}{2}$ so x $\in \{\frac{1 - \sqrt{5}}{2}, \frac{1 + \sqrt{5}}{2}\}$.  General solution: $f(n) = C*(\frac{1 - \sqrt{5}}{2})^{n} + D*(\frac{1 + \sqrt{5}}{2})^{n}$  \par\vspace{0.5cm}

\noindent Now we find the values of $C$ and $D$, given $f(0)$ and $f(1)$: \par\noindent

\noindent $f(0) = C*(\frac{1 - \sqrt{5}}{2})^{0} + D*(\frac{1 + \sqrt{5}}{2})^{0} = C*1 + D*1 = C + D = 0$.  This means $C = -D$ \par

\noindent $f(1) = C*(\frac{1 - \sqrt{5}}{2})^{1} + D*(\frac{1 + \sqrt{5}}{2})^{1} = C*(\frac{1 - \sqrt{5}}{2})^{1} - C*(\frac{1 + \sqrt{5}}{2})^{1} = 1$ \par

\noindent Factoring out C from the left side, we get $C(\frac{1 - \sqrt{5}}{2} - \frac{1 + \sqrt{5}}{2}) = \frac{C}{2}(1 - \sqrt{5})(1 + \sqrt{5}) = 1$ \par

\noindent So $C = \frac{2}{(1 - \sqrt{5})(1 + \sqrt{5})} = \frac{2}{-4} = -\frac{1}{2}$.  Since $C = -D$, $D = -C = -(-\frac{1}{2}) = \frac{1}{2}$

\section{Solve the LHSOR.  Where $n \in N$, $a_{n} = 2$ if $n = 0$, $7$ if $n = 1$, and $a_{n - 1} + 2a_{n - 2}$ otherwise}

$a_{n} = a_{n - 1} + 2a_{n - 2}$ when $a \neq 0$ and $a \neq 1$.  Recurrence relation: $x^{2} = x + 2$. \par
\noindent Rearranging the above equation we get $x^{2} - x - 2 = 0$.  This factors to $(x - 2)(x + 1) = 0$ so $x \in \{-1, 2\}$ \par

\noindent General solution: $a_{n} = C*(-1)^{n} + D*2^{n}$.  Find $C$ and $D$, given $a_{0}$ and $a_{1}$ \par
\noindent $a_{0} = C*(-1)^{0} + D*2^{0} = 2$, $C + D = 2$ \par
\noindent $a_{1} = C*(-1)^{1} + D*2^{1} = 7$, $-C + 2D = 7$ \par
\noindent We have a system of two equations, $C + D = 2$ and $-C + 2D = 7$.  Adding the two equations to get rid of $C$, we get $3D = 9$, thus $D = 3$
\end{document}
