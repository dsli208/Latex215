\documentclass{article}
\usepackage[utf8]{inputenc}
\usepackage{geometry}
\usepackage{amsmath}
\usepackage{physics}
\usepackage{graphicx}
\usepackage{textcomp}
\usepackage{hyperref}
\geometry{legalpaper, portrait, margin = 0.5in}
\rmfamily

\title{AMS 261 HW10 (Hard)}
\author{David S. Li (SBUID: 110328771)}
\date{November 15, 2018}

\begin{document}

\maketitle

\section{15.1.78 - A cross section of Earth's magnetic field can be represented as a vector field in which the center of Earth is located at the origin and the positive $y$-axis points in the direction of the magnetic north pole.  The equation for this field is $F(x, y) = M(x, y)\textbf{i} + N(x, y)\textbf{j} = \frac{m}{(x^{2} + y^{2})^{\frac{5}{2}}}[3xy\textbf{i} + (2y^{2} - x^{2})\textbf{j}]$ where $m$ is the magnetic moment of Earth.  Show that this vector field is conservative.}

\par\noindent\Large From Theorem 15.1 in the textbook, $\frac{\partial N}{\partial x} = \frac{\partial M}{\partial y} \rightarrow F(x, y)$ is a conservative vector field.\vspace{0.25cm}

\par\noindent\Large After being multiplied out, the \textbf{i}-component of $F$ is $M(x, y) = \frac{3xym}{(x^{2} + y^{2})^{\frac{5}{2}}}$.
\par\noindent\huge $\frac{\partial M}{\partial y} = \frac{(x^{2} + y^{2})^{\frac{5}{2}}3xm - 3xym(\frac{5}{2})(x^{2} + y^{2})^{\frac{3}{2}}(2y)}{(x^{2} + y^{2})^{5}} = (3xm)\frac{(x^{2} + y^{2}) - 5y^{2}}{(x^{2} + y^{2})^{\frac{7}{2}}} = \linebreak (3mx)\frac{x^{2} - 4y^{2}}{(x^{2} + y^{2})^{\frac{7}{2}}}$\vspace{0.25cm}

\par\noindent\Large Similarly, the \textbf{j}-component of $F$ is $N(x, y) =$\huge $\frac{2y^{2}m - x^{2}m}{(x^{2} + y^{2})^{\frac{5}{2}}}$
\par\noindent\huge $\frac{\partial N}{\partial x} = \frac{(x^{2} + y^{2})^{\frac{5}{2}}(-2xm) - (2y^{2}m - x^{2}m)(\frac{5}{2})(x^{2} + y^{2})^{\frac{3}{2}}(2x)}{(x^{2} + y^{2})^{5}} = (mx)\frac{(x^{2} + y^{2})(-2) - (2y^{2} - x^{2})(5)}{(x^{2} + y^{2})^{\frac{7}{2}}}\linebreak = (mx)\frac{-2x^{2} - 2y^{2} - 10y^{2} + 5x^{2}}{(x^{2} + y^{2})^{\frac{7}{2}}} = (mx)\frac{3x^{2} - 12y^{2}}{(x^{2} + y^{2})^{\frac{7}{2}}} = (3mx)\frac{x^{2} - 4y^{2}}{(x^{2} + y^{2})^{\frac{7}{2}}}$\vspace{0.25cm}

\par\noindent\Large Therefore, we can conclude $\frac{\partial N}{\partial x} = \frac{\partial M}{\partial y}$ and that $F(x, y)$ is a conservative vector field.

\section{15.2.86 - Determine whether the statement is true or false.  If it is false, explain why or give an example that shows it is false: If $C_{2} = -C_{1}$, then $\int_{C_{1}}f(x, y)ds + \int_{C_{2}}f(x, y)ds = 0$}

\par\noindent\Large Remember that $\int_{C}f(x, y)ds$ is the evaluation of the length of curve $C$.  If $C_{2} = -C_{1}$, that essentially means that $C_{2}$ is opposite an axis from $C_{1}$ but is otherwise identical to $C_{1}$ (i.e., $C_{1}$ could be the upper half of a parabola and $C_{2}$ could be the lower half).\vspace{0.25cm}

\par\noindent\Large Based on this information, $\int_{C_{1}}f(x, y)ds$ and $\int_{C_{2}}f(x, y)ds$ do \textbf{not} cancel each other out, and since they are basically the same length, the statement is \textbf{false}; rather, $\int_{C_{1}}f(x, y)ds + \int_{C_{2}}f(x, y)ds = 2L$, where $L$ is the length of $C_{1}$, and in this case, $C_{2}$ as well.

\section{12.5.2 - Let $r(t)$ be a space curve.  How can you determine whether $t$ is the arc length parameter?}

\par\noindent\Large Remember that the arc length function is $s(t) = \int_{a}^{t}\norm{r'(u)}du$.  For $t$ itself to be the arc length parameter (i.e. $t = s(t)$), there are \textbf{two} conditions: $\norm{r'(u)} = 0$ and $a = 0$ (this would lead to $\int_{0}^{t}du = [u]_{0}^{t} = t - 0 = t$)

\end{document}
