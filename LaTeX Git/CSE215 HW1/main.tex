\documentclass{article}
\usepackage[utf8]{inputenc}
\usepackage{amssymb}
\title{CSE215 HW1}
\author{David S. Li (110328771)}
\date{September 21, 2016}

\begin{document}

\maketitle

\section{Propositional Logic}

\subsection{Logic statements, connectives, etc.}

1. (n $\wedge$ $\neg$ k) $\vee$ (k $\wedge$ $\neg$ n)

2. a) h $\wedge$ (w$\textsubscript{1}$ $\wedge$ w$\textsubscript{2}$) \par\noindent

   b) h $\wedge$ ($\neg$w$\textsubscript{1}$ $\wedge$ $\neg$w$\textsubscript{2}$) \par\noindent
   
   c) ($\neg$h $\wedge$ $\neg$w$\textsubscript{1}$) $\wedge$ w$\textsubscript{2}$ \par\noindent
   
   d) $\neg$w$\textsubscript{2}$ $\vee$ (h $\vee$ w$\textsubscript{1}$) \par\noindent
   
   e) w$\textsubscript{2}$ $\vee$ ($\neg$h $\wedge$ $\neg$w$\textsubscript{1}$) \par\noindent
   
   f) $\neg$w$\textsubscript{2}$ $\vee$ (w$\textsubscript{2}$ $\wedge$ h) \par\noindent
   
   g) w$\textsubscript{2}$ $\vee$ $\neg$h \par\noindent
   
   
\subsection{Truth tables}

    a) Let our implication be p $\Rightarrow$ q\par
    
    \begin{center}
    \begin{tabular}{||c c c c||}
    \hline
    p & q & p $\Rightarrow$ q & q $\Rightarrow$ p \\
    \hline \hline
    
         T & T & T & T \\
         \hline
         F & T & T & F \\ 
         \hline
         T & F & F & T \\
         \hline
         F & F & T & T \\
         \hline
         
         
    \end{tabular}
    \end{center}
    
    As shown above, p $\Rightarrow$ q and its converse are NOT EQUAL.\par\par
    
    \noindent
    b)
    
    \begin{center}
    \begin{tabular} {|| c c c c c c c c ||}
    \hline
    p & q & r & (p $\vee$ q) & (p $\vee$ q) $\Rightarrow$ r & p $\Rightarrow$ r & q $\Rightarrow$ r & (p $\Rightarrow$ r) $\wedge$ (q $\Rightarrow$ r) \\
    \hline \hline
    
    T & T & T & T & T & T & T & T \\
    \hline
    T & T & F & T & F & F & F & F \\
    \hline
    T & F & T & T & T & T & T & T \\
    \hline
    T & F & F & T & F & F & T & F \\
    \hline
    F & T & T & T & T & T & T & T \\
    \hline
    F & T & F & T & F & T & F & F \\
    \hline
    F & F & T & F & T & T & T & T \\
    \hline
    F & F & F & F & T & T & T & T \\
    \hline
    \end{tabular}
    \end{center}
    
    As shown in the table above, p $\vee$ q $\Rightarrow$ r and (p $\Rightarrow$ r) $\wedge$ (q $\Rightarrow$ r) are logically equivalent. \par\par
    
    
\subsection{Showing valid arguments}
    1) Let p be "Daniel solves the problem correctly" and q be "Daniel gets 2 more points" \par
    
    p $\Rightarrow$ q \par
    q \par
    $\therefore$ p \par\par
    \noindent
    
    2) Let p be "I start watching Game of Thrones", q be "I will not be able to study", and r be "I will fail the course" \par
    p $\Rightarrow$ q \par
    q $\Rightarrow$ r \par
    
    $\therefore$ p $\Rightarrow$ r \par\par
    \noindent
    3) Let p be "Samantha knows Java" and q be "Samantha knows C++"\par
    p $\wedge$ q \par
    q \par
    $\therefore$ p \par
    \noindent
    4) Let p be "I do yoga" and q be "I am healthy"\par
    p $\Rightarrow$ q\par
    $\neg$q \par
    $\therefore$ $\neg$ p
    \noindent
    5) Let p be "I threw the ball in the air", q be "I threw the ball off the cliff", and r be "The ball drops"\par
    p $\Rightarrow$ r\par
    q $\Rightarrow$ r\par
    $\therefore$ r\par
    \noindent
    
\section{Predicate Calculus}

\subsection{}
    The truth set can be denoted as $\{$(x, y) $\in$ $\mathbb{R}$ $\mid$ $\mid$x$\mid$  \textless  $\mid$y$\mid$$\}$ \par\par
    \noindent
    \noindent
    $\mathbb{R}$ includes negative numbers, so we use absolute values for x and y so if y were more negative compared to x, Q(x, y) would still return true.  For positive values, a value x \textless y would also return true.
\subsection{}
    Set m = 27 and n = 9 (so n$\textsuperscript{2}$ = 81)\par
    \noindent
    m would be a factor of n$\textsuperscript{2}$, as 27 * 3 = 81.  However m, does not divide evenly into $n$ (9 / 27 = $\frac{1}{3}$) so the below statement is true\par
    \noindent
    $\exists$ (m, n) $\in$ Z such that $\neg$P(m, n)
\subsection{Counterexamples}
    a) Assume x = 1 and plug that value in for x\par
    \noindent
    1 $\ngtr$ $\frac{1}{1}$, since those two are EQUAL, but one is not greater than the other\par
    
    \noindent
    b) Assume m = 2, n = 4\par
    \noindent
    2.4 $\ngeq$ 2 + 4 so 2.4 $\ngeq$ 6
    \noindent
    c) Assume x = 3\par
    \noindent
    (3)$\textsuperscript{2}$ = 9 $\neq$ 2\par\noindent
    *Note that the square root of 2 is an irrational number as it is not a perfect square, so that can automatically imply there is no solution*\par\noindent
    d)Assume x = 10 and y = 5\par\noindent
    10 \textgreater  5, however 10 - 5 $\nless$ 5 (10 - 5 = 5 instead) \par\noindent
    e) 2 is a prime number, but unlike other prime numbers, it is even, making the statement false
    
\subsection{Formal symbolic statements}
    a) Let N be the set of things that can go up and P(x) mean coming down\par\noindent
    $\forall$x $\in$ N, P(x)\par\noindent
    b) Let A be the set of objects for this problem and P(x, y) be the function where x likes y\par\noindent
    $\forall$ x $\in$ A, $\exists$ y $\in$ A such that P(x, y) $\wedge$ $\neg$P(y, x) $\wedge$ $\exists$ z such that P(z, x)\par\noindent
\end{document}
