\documentclass{article}
\usepackage[utf8]{inputenc}
\usepackage{geometry}
\usepackage{amsmath}
\usepackage{physics}
\geometry{legalpaper, portrait, margin = 0.5in}
\rmfamily

\title{AMS 261 Homework 6}
\author{David S. Li (SBUID: 110328771)}
\date{October 3, 2018}

\begin{document}

\maketitle

\section{Textbook Problems}
\subsection{13.3.104 - Show that the function $z = e^{-t}cos(\frac{x}{c})$ satisfies the heat equation $\frac{\partial z}{\partial t} = c^{2}\frac{\partial^{2}z}{\partial x^{2}}$}

\par\noindent\large $\frac{\partial z}{\partial t} = -e^{-t}cos(\frac{x}{c})$, $\frac{\partial z}{\partial x} = -\frac{e^{-t}}{c}sin(\frac{x}{c})$, and $\frac{\partial^{2}z}{\partial x^{2}} = -\frac{e^{-t}}{c^{2}}cos(\frac{x}{c})$\vspace{0.25cm}

\par\noindent\large By setting $\frac{\partial z}{\partial t} = u\frac{\partial^{2}z}{\partial x^{2}}$, we can see that to get from the right-hand side to the left-hand side, all we need to do is multiply the right hand side by $c^{2}$.  Therefore, the heat equation is satisfied.

\subsection{13.3.108 - Determine whether there exists a function $f(x, y)$ with the given partial derivatives: $f_{x}(x, y) = 2x + y$, $f_{y}(x, y) = x - 4y$.  Explain your reasoning.  If such a function exists, give an example.}

\par\noindent\large The partial derivative with respect to $x$ is taking the derivative of the original function (if it exists) to $x$, and the same can be said for $y$.  So to find if the original function does exist, we can take the \textbf{antiderivative} of the partial derivatives of $x$ and $y$ and see if we can reach a common antiderivative.\vspace{0.25cm}

\par\noindent\large $\int f_{x}(x, y) = \int (2x + y)dx = x^{2} + xy$
\par\noindent\large $\int f_{y}(x, y) = \int (x - 4y)dy = xy - 2y^{2}$\vspace{0.25cm}

\par\noindent\large Since $x^{2} + xy \neq xy - 2y^{2}$, there does \textbf{not} exist a function $f(x, y)$ with the above partial derivatives.

\subsection{13.5.48 - The path of an object represented by $w = f(x, y)$ is shown, where $x$ and $y$ are functions of $t$.  The point on the graph represents the position of the object.  Determine whether each of the following is positive, negative, or zero:}
\subsubsection{(a)$\frac{dx}{dt}$}
\par\noindent\large Since the particle is turning leftwards and eventually downwards at the point on the graph, $\frac{dx}{dt}$ is \textbf{negative}.
\subsubsection{(b)$\frac{dy}{dt}$}
\par\noindent\large Since the particle is not at the highest point yet, $\frac{dy}{dt}$ is still \textbf{positive} (after the highest point, $\frac{dx}{dt}$ will become negative).

\section{PDF Problems}
\subsection{The price $P$ (in dollars) of a used car is a function of its initial cost $C$ (in dollars) and its age $A$ (in
years). What are the units of $\partial P/\partial A$? Is $\partial P/\partial A$ positive or negative? Explain.}
\par\noindent\large  In real life, the price of a car will decrease with value.  Assuming that situation, $\partial P/\partial A$ will be of the units $\frac{dollars}{year}$, as it is denoting by how much the car decreases in value (in dollars) each year.  Since the car value \textbf{decrases} each year, the partial derivative will be \textbf{negative}.

\subsection{Let $f(x,y,z) = 2x + 3xy - 5yz^{2} + 2 sin(az) + bxyz + a cos(b)$, where $f$ is a function of $x$, $y$, and $z$, with constants $a$ and $b$. Find the total differential of $f$.}
\par\noindent\Large Total differential of a function $df(x, y, z) = \frac{\partial f}{\partial x}dx + \frac{\partial f}{\partial y}dy + \frac{\partial f}{\partial z}dz$
\par\noindent\Large $\frac{\partial f}{\partial x} = 2 + 3y + byz$, $\frac{\partial f}{\partial y} = 3x - 5z^{2} + bxz$, and $\frac{\partial f}{\partial z} = -10yz + 2acos(az)$
\par\noindent\large Therefore, we get as our total differential $df =  (2 + 3y + byz)dx + (3x - 5z^{2} + bxz)dy + (-10yz + 2acos(az))dz$
\end{document}
