\documentclass{article}
\usepackage[utf8]{inputenc}
\usepackage{geometry}
\usepackage{amsmath}
\usepackage{physics}
\geometry{legalpaper, portrait, margin = 0.5in}
\rmfamily

\title{AMS 261 Chapter 13 Challenge Problem 1}
\author{David S. Li (SBUID: 110328771)}
\date{October 17, 2018}

\begin{document}

\maketitle

\section{Consider a mountainous landscape described by the function $f(x, y) = 5000[1 + cos(\frac{2x + y}{300}\pi)cos(\frac{x + 2y}{200}\pi)]$, where $x$ and $y$ are positions in miles and $f(x, y)$ is elevation (in feet).}
\subsection{If a hiker is standing at the position $(x, y) = (30, -70)$, what is the direction of steepest ascent up the mountain?}

% $\norm{\nabla f(x, y)}$ is RATE of maximum increase
\par\noindent\Large Let $g(x, y) = cos(\frac{2x + y}{300}\pi)$ and $h(x, y) = cos(\frac{x + 2y}{200}\pi)$. Our equation will then become $f(x, y) = 5000[1 + g(x, y)h(x, y)]$.  Direction of steepest ascent (\textbf{maximum increase}) is equivalent to $\nabla f(x, y) = f_{x}\textbf{i} + f_{y}\textbf{j}$.\vspace{0.25cm}

\par\noindent\Large $f_{x} = 5000 * \frac{\partial}{\partial x}[1 + g(x, y)h(x, y)] = 5000[g(x, y)h_{x}(x, y) + g_{x}(x, y)h(x, y)]$
\par\noindent\Large $g_{x} = -sin(\frac{2x + y}{300}\pi)(\frac{(300)(2)}{90,000}\pi) = -\frac{\pi}{150}sin(\frac{2x + y}{300}\pi)$, $h_{x} = -sin(\frac{x + 2y}{200}\pi)(\frac{(200)(1)}{40,000}\pi)= -\frac{\pi}{200}sin(\frac{x + 2y}{200}\pi)$\vspace{0.25cm}

\par\noindent\Large $f_{y} = 5000 * \frac{\partial}{\partial y}[1 + g(x, y)h(x, y)] = 5000[g(x, y)h_{y}(x, y) + g_{y}(x, y)h(x, y)]$
\par\noindent\Large $g_{y} = -sin(\frac{2x + y}{300}\pi)(\frac{(300)(1)}{90,000}\pi) = -\frac{\pi}{300}sin(\frac{2x + y}{300}\pi)$, $h_{y} = -sin(\frac{x + 2y}{200}\pi)(\frac{200(2)}{40,000}\pi) = -\frac{\pi}{100}sin(\frac{x + 2y}{200}\pi)$\vspace{0.25cm}

\par\noindent\Large $g(30, -70) = cos(\frac{2(30) - 70}{300}\pi) = cos(-\frac{\pi}{30})$, $h(30, -70) = cos(\frac{30 + 2(-70)}{200}\pi) = cos(-\frac{11\pi}{20})$
\par\noindent\Large $g_{x}(30, -70) = -\frac{\pi}{150}sin(-\frac{\pi}{30})$, $h_{x}(30, -70) = -\frac{\pi}{200}sin(-\frac{11\pi}{20})$, $g_{y}(30, -70) = -\frac{\pi}{300}sin(-\frac{\pi}{30})$,
\par\noindent\Large $h_{y}(30, -70) = -\frac{\pi}{100}sin(-\frac{11\pi}{20})$\vspace{0.25cm}

% Wolfram formula: 5000[(cos(-pi/30))(-pi/100)(sin(-11pi/20))+ (-pi/300)(sin(-pi/30))(cos(-11pi/20))]
\par\noindent\Large $f_{x}(30, -70) = 5000[(cos(-\frac{\pi}{30}))(-\frac{\pi}{200}sin(-\frac{11\pi}{20})) + (-\frac{\pi}{150}sin(-\frac{\pi}{30}))(cos(-\frac{11\pi}{20}))] \approx 75.44$
\par\noindent\Large $f_{y}(30, -70) = 5000[(cos(-\frac{\pi}{30}))(-\frac{\pi}{100}sin(-\frac{11\pi}{20})) + (-\frac{\pi}{300}sin(-\frac{\pi}{30}))(cos(-\frac{11\pi}{20}))] \approx 153.4396$\vspace{0.25cm}

\par\noindent\Large So we get $\nabla f(x, y) = f_{x}(30, -70)\textbf{i} + f_{y}(30, -70)\textbf{j} = 75.44\textbf{i} + 153.4396\textbf{j}$ as our maximum increase
%\par\noindent\Large Maximum increase = $\norm{\nabla f(x, y)} = \sqrt{(75.44)^{2} + (153.4396)^{2}} \approx 170.982$

\subsection{If the hiker were to travel in this direction, what is the degree of incline?  Remember that $x$ and $y$ are in \textit{miles} whereas $f(x, y)$ is in \textit{feet}.}
% Magnitude?
\par\noindent\Large To get the degree of incline $\theta$, we need to know two things, our distance in the $xy$ plane, and our value of $f(x, y)$ given that distance.  Given the gradient of our vector (assuming we go in the direction of maximum increase as specified in part a), we can take the magnitude of the gradient to get a slope of the mountain's incline in that direction: 
\par\noindent\Large $\norm{\nabla f(30, -70)} = \sqrt{(75.44)^{2} + (153.4396)^{2}} \approx 170.982$ feet per mile.\vspace{0.25cm}

\par\noindent\Large This value of the directional derivative is the slope of the mountain per mile.  This can be represented as $D_{u}f(30, -70) = \nabla f(30, -70) \cdot u$\vspace{0.25cm}%= 170.982$ feet per mile (as the unit vector would go in the same direction).
%\par\noindent\Large To get our height component, simply evaluate $f(30, -70)$: $\linebreak f(30, -70) = 5000[1 + cos(\frac{2(30) - 70}{300}\pi)]= 5000[1 + cos(-\frac{\pi}{30}) \approx 9972.61$ feet, or $1.89$ mi.\vspace{0.25cm}

%\par\noindent\Large Finally, to get the degree of incline $\theta$, take $tan^{-1}(\frac{1.89}{170.982}) = 0.63$ degrees.


% DIRECTIONAL DERIVATIVE

%\par\noindent\Large To get the degree of incline $\theta$, we can take the directional derivative of $f(x, y)$, given that $D_{u}f(x, y) = f_{x}(x, y)cos(\theta) + f_{y}(x, y)sin(\theta) = \nabla f \cdot u$.\vspace{0.25cm}

\par\noindent\Large Now, we need to find a unit vector $u$ that will match with the gradient that we found in part a.  Since the gradient represents the direction that the climber would ideally go in, we can use \textit{that} as our vector to get our next point.\vspace{0.25cm}

\par\noindent\Large $\textbf{u} = \frac{\nabla f}{\norm{\nabla f}}$, where $\norm{\nabla f(x, y)} = \sqrt{(75.44)^{2} + (153.4396)^{2}} \approx 170.982$, so our unit vector is $\textbf{u} = \frac{75.44}{170.892}\textbf{i} + \frac{153.4396}{170.892}\textbf{j} \approx 0.44\textbf{i} + 0.897\textbf{j}$\vspace{0.25cm}

\par\noindent\Large So $D_{u}f(x, y) = \nabla f \cdot u = (75.44\textbf{i} + 153.4396\textbf{j}) \cdot (0.44\textbf{i} + 0.897\textbf{j}) = (75.44)(0.44) + (153.4396)(0.897) = 170.82$ feet per mile - \textbf{approximately the same numerical value as the magnitude of our gradient}, meaning $D_{u}f(x, y) = \norm{\nabla f(x, y)}$ in this case.\vspace{0.25cm}

%\par\noindent\Large Now, to get our value of $\theta$, plug this value into the equation for the other expression for the value of the directional derivative $D_{u}f(x, y)$: $171.07 = f_{x}(30, -70)cos(\theta) + f_{y}(x, y)sin(\theta)$.
%\par\noindent\Large By plugging our values from part 1, we get $D_{u}f(30, -70) = 171.07 = 75.44cos(\theta) + 153.44sin(\theta)$

\par\noindent\Large Since the units are feet per mile, we rearrange $D_{u}f$ as $\frac{170.82}{5280} = 0.32354$.  Taking the inverse tan of this result of our "rise over run" gives us an incline of \textbf{1.85 degrees}.

\subsection{Suppose the hiker is a bit tired and doesn’t want to travel in a direction that has a degree of incline higher than 0.75 degree. In what direction should the hiker travel to ascend the mountain most rapidly (subject to this constraint)?}

\par\noindent\Large Assuming 0.75 is the maximum desired incline, we can take $tan(75)$ to get a value of $D_{u}f$ that reflects this: $tan(0.75) \approx 0.0130907$.  Remembering that we want to get it in feet per miles, $0.0130907 * 5280 \approx 69.12$ feet per mile.

\par\noindent\Large Assuming the same conditions as in the previous parts, $D_{u}f(30, -70) = \norm{\nabla f(30, -70)}$ (as the computed value of $D_{u}f(x, y)$ was extremely close to $\norm{\nabla f(x, y)}$, and because $f$ and $u$ are going in the same direction, we can assume that the two values will be equal).\vspace{0.25cm}

\par\noindent\Large This gives us $\norm{\nabla f(30, -70)} = 69.12 = \sqrt{(x_{i})^{2} + (y_{j})^{2}}$, where $x_{i}$ and $y_{j}$ are the coefficients for the gradient vector at incline 0.75 degrees.
\par\noindent\Large The equation becomes $4777.42 = (x_{i})^{2} + (y_{j})^{2}$ - a possible constraint pertaining to our original function $f(x, y)$.  Because of this, we get $g(x, y) = (x_{i})^{2} + (y_{j})^{2} - 4777.42$ and can now use Lagrange Multipliers to obtain a solution for the two equations.\vspace{0.25cm}

\par\noindent\Large So $\nabla f(x, y) = \lambda\nabla g(x, y)$, or $\nabla f(30, -70) = \lambda\nabla g(30, -70)$.  Taking partial derivatives and multiplying in $\lambda$ then gives us $75.44\textbf{i} + 153.4396\textbf{j} = 2x_{i}\lambda\textbf{i} + 2y_{j}\lambda\textbf{j}$.  From here we can get two equations: $2x_{i}\lambda = 75.44$ and $2y_{j}\lambda = 153.4396$ which combined with our constraint equation $4777.42 = (x_{i})^{2} + (y_{j})^{2}$ gives us a system of equations to solve for $x_{i}$ and $y_{j}$.\vspace{0.25cm}

\par\noindent\Large From the first of the three equations, we get $\lambda = \frac{75.44}{2x_{i}}$.  Plugging into the second equation, we get $2y_{j}(\frac{75.44}{2x_{i}}) = 75.44\frac{y_{j}}{x_{i}} = 153.4396$.  Rearranging, this becomes approx. $y_{j} = 2.034x_{i}$.\vspace{0.25cm}

\par\noindent\Large Substituting back into the constraint, we get $4777.42 = (x_{i})^{2} + (2.034x_{i})^{2} = 5.137x_{i}^{2}$.  Solving algebraically, we get $x_{i}^{2} \approx 1,154.76$ and $x_{i} = 33.98$.  Plugging into our equation for $y_{j}$ gives us $y_{j} = 2.034(33.98) \approx 69.12$, giving us a direction $\nabla f(30, -70) = 33.98\textbf{i} + 69.12\textbf{j}$.\vspace{0.25cm}

\par\noindent\large (Note that even though the square root is supposed to produce both a positive and a negative, taking either value would produce the same vector - if we took the negative square root, we would get vector $\nabla f(30, -70) = -33.98\textbf{i} - 69.12\textbf{j}$, the same vector as its positive counterpart).
\end{document}
