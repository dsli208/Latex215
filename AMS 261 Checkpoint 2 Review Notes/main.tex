\documentclass{article}
\usepackage[utf8]{inputenc}
\usepackage{geometry}
\usepackage{amsmath}
\usepackage{physics}
\usepackage{graphicx}
\geometry{legalpaper, portrait, margin = 0.5in}
\rmfamily

\title{AMS 261 Checkpoint 2 Review Notes}
\author{David Li}
\date{October 2018}

\begin{document}

\maketitle

\section{Textbook Problems (not including examples)}
\subsection{13.5.53 - Given the functions $u(x, y)$ and $v(x, y)$, verify that the Cauchy-Riemann equations $\frac{\partial u}{\partial x} = \frac{\partial v}{\partial y}$ and $\frac{\partial u}{\partial y} = -\frac{\partial u}{\partial x}$ can be written in polar coordinate form as $\frac{\partial u}{\partial r} = \frac{1}{r} \cdot \frac{\partial v}{\partial\theta}$ and $\frac{\partial v}{\partial r} = -\frac{1}{r}\frac{\partial u}{\partial\theta}$}

\par\noindent\Large Convert to \textbf{polar} using $x = rcos(\theta)$ and $y = rsin(\theta)$ $\frac{\partial u}{\partial r} = \frac{\partial u}{\partial x}\frac{\partial x}{\partial r} + \frac{\partial u}{\partial y}\frac{\partial y}{\partial r} = u_{x}cos(\theta) + u_{y}sin(\theta)$\vspace{0.25cm}

\par\noindent\Large Similarly, $\frac{\partial v}{\partial r} = \frac{\partial v}{\partial x}\frac{\partial x}{\partial r} + \frac{\partial v}{\partial y}\frac{\partial y}{\partial r} = v_{x}cos(\theta) + v_{y}sin(\theta)$

\subsection{13.6.55 - The temperature at the point $(x, y)$ on a metal plate is $T(x, y) = \frac{x}{x^{2} + y^{2}}$.  Find the direction of greatest increase in heat from the point $(3, 4)$.}

\par\noindent\large Let the path be represented by position vector $r(t) = x(t)\textbf{i} + y(t)\textbf{j}$.  Deriving, we get $r'(t) = \frac{dx}{dt}\textbf{i} + \frac{dy}{dy}\textbf{j}$.  
\par\noindent\Large This is equivalent to the gradient $\nabla T = \frac{y^{2} - x^{2}}{(x^{2} + y^{2})^{2}}\textbf{i} - \frac{2xy}{(x^{2} + y^{2})^{2}}\textbf{j}$ because we want \textbf{maximum increase}.\vspace{0.25cm}

\par\noindent\Large So we get $\frac{y^{2} - x^{2}}{(x^{2} + y^{2})^{2}} = k\frac{dx}{dt}$ and $\frac{2xy}{(x^{2} + y^{2})^{2}}= k\frac{dy}{dt}$, therefore $\frac{y^{2} - x^{2}}{(x^{2} + y^{2})^{2}}\frac{1}{dx} = \frac{k}{dt}$ and $\frac{2xy}{(x^{2} + y^{2})^{2}}\frac{1}{dy} = \frac{k}{dt}$.
\par\noindent\Large This becomes $\frac{y^{2} - x^{2}}{(x^{2} + y^{2})^{2}}\frac{1}{dx} = \frac{2xy}{(x^{2} + y^{2})^{2}}\frac{1}{dy} \rightarrow \frac{y^{2} - x^{2}}{dx} = \frac{2xy}{dy} \rightarrow \frac{dx}{y^{2} - x^{2}} = \frac{dy}{2xy}$ \textbf{(what do I do next?)}

\subsection{13.7.1 - Consider a point $(x_{0}, y_{0}, z_{0})$ on a surface given by $F(x, y, z) = 0$.  What is the relationship between $\nabla F(x_{0}, y_{0}, z_{0})$ and any tangent vector \textbf{v} at $(x_{0}, y_{0}, z_{0})$?  How do you represent this relationship mathematically?}

\par\noindent\large $\nabla F(x_{0}, y_{0}, z_{0})$ is \textbf{orthogonal} to the tangent plane (and therefore all vectors within the tangent plane).  Theorem is listed within Section 3, under Section 13.7.

\subsection{13.8.33 - Determine whether there is a relative maximum, a relative minimum, a saddle point, or insufficient information to determine the nature of the function $f(x, y)$ at the critical point $(x_{0}, y_{0})$: $f_{xx}(x_{0}, y_{0}) = -9$, $f_{yy}(x_{0}, y_{0}) = 6$, $f_{xy}(x_{0}, y_{0}) = 10$ }

\par\noindent\large Use $d = f_{xx}f_{yy} - f_{xy}^{2} = (-9)(6) - (10)^{2} = -54 - 100 = -154 \rightarrow$ \textbf{saddle point}

\subsection{13.PS.6, 13.PS.7 - A heated storage room has the shape of a rectangular prism and hsa a volume of 1000 cubic feet, as shown in the figure.  Because warm air rises, the heat loss per unit of area through the ceiling is five times as great as the heat loss through the floor.  The heat loss through the four walls is three times as great as the heat loss through the floor.}

\subsubsection{13.PS.7 - Determine the room dimensions that will minimize heat loss and therefore minimize heating costs (aside from the floor).}
%\par\noindent\large Let $L$ represent heat loss through the floor.  Heat loss through the walls can be represented as $3L$ and the heat loss through the ceiling can be represented as $5L$.  Total heat loss is therefore $9L$.\vspace{0.25cm}

\par\noindent\large We have two types of surface where heat can be lost: walls where heat loss is $2xz + 2yz$ and the ceiling with heat loss represented by $xy$ (floor is discounted in this part).  Factoring in loss relative to the floor from 13.PS.6, we get total heat loss $L_{T} = f(x, y, z) = 2xz + 2yz + xy$.  This is subject to the constraint of our room dimensions $g(x, y, z) = V = xyz = 1000$ ft\textsuperscript{3}.\vspace{0.25cm}

\par\noindent\large Using Lagrange multipliers, we set $\nabla f(x, y, z) = \lambda\nabla g(x, y, z)$.
\par\noindent\large We get $\nabla f(x, y, z) = (2z + y)\textbf{i} + (2z + x)\textbf{j} + (2x + 2y)\textbf{k}$ and $\nabla g(x, y, z) = yz\textbf{i} + xz\textbf{j} + xy\textbf{k}$.  Therefore, we get the equation $(2z + y)\textbf{i} + (2z + x)\textbf{j} + (2x + 2y)\textbf{k} = yz\lambda\textbf{i} + xz\lambda\textbf{j} + xy\lambda\textbf{k}$.\vspace{0.25cm}

\par\noindent\large This gives us a system of equations: (\textbf{i}) $2z + y = yz\lambda$, (\textbf{j}) $2z + x = xz\lambda$, (\textbf{k}) $2(x + y) = xy\lambda$, and (constraint) $xyz = 1000$. Solving the (\textbf{i}) equation, we get $\lambda = \frac{2}{y} + \frac{1}{z}$.  Plugging into the (\textbf{j}) and (\textbf{k}) equations, we get $2z + x = \frac{2xz}{y} + x$ and $2x + 2y = 2x + \frac{xy}{z}$ respectively.\vspace{0.25cm}

\par\noindent\large Simplifying the (\textbf{j}) equation will give us $2z = \frac{2xz}{y} \rightarrow 1 = \frac{x}{y} \rightarrow x = y \rightarrow y = x$.
\par\noindent\large Simplifying the (\textbf{k}) equation will give us $2y = \frac{xy}{z} \rightarrow 2yz = xy \rightarrow x = 2z \rightarrow z = \frac{1}{2}x$.
\par\noindent\large Plugging the two values of $y$ and $z$ into our constraint equation gives us $(x)(x)(\frac{1}{2}x) = \frac{1}{2}x^{3} = 1000$.
\par\noindent\large This gives us $x^{3} = 2000$.  Taking the cube root of both sides gives us $x = 10\sqrt[3]{2}$.  Now, simply plug into the derived equations to get the values of $y$ and $z$: $y = x = 10\sqrt[3]{2}$ and $z = \frac{1}{2}x = 5\sqrt[3]{2}$

\subsection{14.1.69 - Determine whether each expression represents the area of the shaded region (see figure).}
\subsubsection{$\int_{0}^{5}\int_{y}^{\sqrt{50 - y^{2}}}dydx$}
\par\noindent\large Not valid - doing the first integral would give us $\int_{0}^{5}[y]^{\sqrt{50 - y^{2}}}_{y}dx = \int_{0}^{5}(\sqrt{50 - y^{2}} - y) dx$.  The variable $y$ is still left in when we do the outer integral to integrate in terms of $x$.

\subsubsection{$\int_{0}^{5}\int_{x}^{\sqrt{50 - x^{2}}}dydx$}
\par\noindent\large Yes - as opposed to the previous part, we integrate with respect to the proper variable.

\subsubsection{}

\subsection{14.3.45 - Use a double integral to find the area of the shaded region (given by $r = 2sin(3\theta)$)}
\textbf{ASK ABOUT THIS AND THE 2ND REVIEW PROBLEM IN RECITATION TODAY}

\par\noindent\large \section{Review Problems}
\subsection{1 - Consider the function $f(x, y) = 2x^{2} + sin(\pi y) - x + 4$ in the region $x^{2} + y^{2} \leq 1$}
\subsubsection{(a) Determine the critical points of $f$ in $x^{2} + y^{2} < 1$ and classify each as a maximum, minimum, or saddle point}

\par\noindent\large Remember critical points are at $f_{x} = 0$ and $f_{y} = 0$: $f_{x} = 4x - 1 = 0$ and $f_{y} = \pi cos(\pi y) = 0$.  The only value of $x$ that satisfies $f_{x} = 0$ is $x = \frac{1}{4}$, whereas there are two values where $f_{y} = 0$ is satisfied: $y = \pm\frac{1}{2}$.  This means we have \textbf{two} critical points: $(\frac{1}{4}, -\frac{1}{2})$ and $(\frac{1}{4}, \frac{1}{2})$.

\section{Homework 9 Recommended Problems}

\section{Things to Review/Notes}
\subsection{Section 13.5}
\par\noindent\Large Chain Rule (one or multiple variables...): $\frac{\partial w}{\partial s} = \frac{\partial w}{\partial x}\frac{dx}{ds} + \frac{\partial w}{\partial y}\frac{dy}{ds}$ and $\frac{\partial w}{\partial t} = \frac{\partial w}{\partial x}\frac{dx}{dt} + \frac{\partial w}{\partial t}\frac{dy}{dt}$ and so on...

\par\noindent\Large Chain Rule: Implicit Differentiation for $F(x, y) = 0$ implicitly defining $y$ as a differentiable function of $x$: $\frac{dy}{dx} = -\frac{F_{x}(x, y)}{F_{y}(x, y)}$, $F_{y}(x, y) \neq 0$
\par\noindent\Large Chain Rule: Implicit Differentiation for $F(x, y) = 0$ implicitly defining $z$ as a differentiable function of $x$ and $y$: $\frac{\partial z}{\partial x} = -\frac{F_{x}(x, y, z)}{F_{z}(x, y, z)}$ and $\frac{\partial z}{\partial y} = -\frac{F_{y}(x, y, z)}{F_{z}(x, y, z)}$, where $F_{z}(x, y, z) \neq 0$

\subsection{Section 13.7}
\par\noindent\large Equation of Tangent Plane: $F_{x}(x_{0}, y_{0}, z_{0})(x - x_{0}) + F_{y}(x_{0}, y_{0}, z_{0})(y - y_{0}) + F_{z}(x_{0}, y_{0}, z_{0})(z - z_{0}) = 0$ (derived from the dot product between $\nabla F(x_{0}, y_{0}, z_{0})$ and any vector \textbf{v} in the tangent plane for $\nabla F(x_{0}, y_{0}, z_{0})$ being equal to 0 due to the two being orthogonal).

\subsection{Section 13.8}
\par\noindent\large Extreme Value Theorem: There is at least one point in region $R$ where $f$ takes on a minimum value, and at least one point where $f$ takes on a maximum
\par\noindent\large Relative minimum at $(x_{0}, y_{0})$ if $f(x, y) \geq f(x_{0}, y_{0})$, relative maximum at $(x_{0}, y_{0})$ if $f(x, y) \leq f(x_{0}, y_{0})$\vspace{0.25cm}

\par\noindent\large Critical point if $f_{x}(x_{0}, y_{0}) = 0$ \textbf{and} $f_{y}(x_{0}, y_{0})$ \textbf{or} $f_{x}(x_{0}, y_{0})$ or $f_{y}(x_{0}, y_{0})$ DNE.\vspace{0.25cm}

\par\noindent\large Second Partials Test: See \textbf{Formula Sheet}

\section{Section 13.9}
\par\noindent\large Least Square Regression Line for ${(x_{1}, y_{1}), (x_{2}, y_{2}), ... (x_{n}, y_{n})}$ is in the linear equation $f(x) = ax + b$ where:
\par\noindent\huge $a = \frac{n\Sigma_{i = 1}^{n}x_{i}y_{i} - \Sigma_{i = 1}^{n}x_{i}\Sigma_{i = 1}^{n}y_{i}}{\Sigma_{i = 1}^{n}x_{i}^{2} - (\Sigma_{i = 1}^{n}x_{i})^{2}}$ and $b = \frac{1}{n}(\Sigma_{i = 1}^{n}y_{i} - a\Sigma_{i = 1}^{n}x_{i})$

\section{Section 13.10}
\par\noindent\large Lagrange multiplier equation is dependent on number of constraints:
\par\noindent\large One constraint: $\nabla f(x, y) = \lambda\nabla g(x, y)$, two constraints: $\nabla f(x, y) = \lambda\nabla g(x, y) + \mu\nabla h(x, y)$

\end{document}
