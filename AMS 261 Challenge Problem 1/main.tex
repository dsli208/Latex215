\documentclass{article}
\usepackage[utf8]{inputenc}
\usepackage{geometry}
\usepackage{amsmath}
\usepackage{physics}
\geometry{legalpaper, portrait, margin = 0.5in}
\rmfamily

\title{AMS 261 Challenge Problem 1}
\author{David S. Li (SBUID: 110328771)}
\date{September 22, 2018}

\begin{document}

\maketitle

\section{A restaurant has a pedestal for displaying the day’s desserts to customers. Both the base and top of the pedestal are level and square-shaped. Each side of the base is 6 ft. long and each side of the top is 4 ft. The top is 3 ft. above the base. Note that the squares of the top and base are similarly aligned, similar to the grain chute in problem 106 of section 11.5.}

\subsection{Determine the angle between the top and each side.}

\par\noindent\large To get the angle between each side of the pedestal and the top, assuming the center of the pedestal is on the same line as that of the top, we take the distance from the side of the pedestal to just under the top plate to get a value of $a = 2$.\vspace{0.25cm}
\par\noindent\large Given the height $h = 3$, so using the pythagorean theorem we would get $c^{2} = a^{2} + h^{2}$, where $c$ is the hypotenuse formed from half the length of one edge and the height.  Solving, we get $c^{2} = (2)^{2} + (3)^{2} = 13$, so $c = \sqrt{13}$.
\par\vspace{0.25cm}\noindent\large After this, we let $\theta$ serve as the angle for the triangle where the hypotenuse runs from an edge on the top to a side on the pedestal that shares the height from the pedestal to the top.  To get the angle, we set $tan(\theta) = \frac{2}{3}$.  Taking the inverse tangent, we get $\theta = tan^{-1}(\frac{2}{3}) = 33.69$ degrees.

\subsection{Determine the angle between two adjacent sides.}

\par\noindent\large We can treat two adjacent sides as two planes and get the vectors.  Using the top of the center of the pedestal as our origin, and using the two sloped sides immediately to the left and the bottom of the top of the pedestal as our reference planes, we get two reference points to get our vectors in the two planes, denoted as $P_{1}$ for the left plane and $P_{2}$ for the bottom plane. \vspace{0.25cm}

\par\noindent\large Our points are for $P_{1}$ $(-3, 0, 0)$ (left pedestal reference) and $(-2, 0, 3)$ (left top reference) and for $P_{2}$ $(0, -3, 0)$ (bottom pedestal reference) and $(0, -2, 3)$ (bottom top reference).  The left reference points are on the left edge of each square, but equidistant from the top and bottom edges; the same concept applies for the bottom reference points, where they are on the bottom edge, but equidistant from the left and right edges.\vspace{0.25cm}.  We also will use a point location on the line of intersection of both sides, $p_{b} = (-3, -3, 0)$

%\par\noindent\large Vector $v_{1}$ for plane $P_{1}$ is therefore $<-2 - (-3), 0 - 0, 4 - 0> = <1, 0, 4>$.  Vector $v_{2}$ for plane $P_{2}$ is $<0 - 0, -2 - (-3), 4 - 0> = <0, 1, 4>$.
\par\noindent\large For plane $P_{1}$, vector $u_{1}$ is therefore $<-2 - (-3), 0 - 0, 3 - 0> = <1, 0, 3>$ and vector $v_{1}$ is\\ $<-3 - (-3), -3 - 0, 0 - 0> = <0, -3, 0>$.  For plane $P_{2}$, vector $u_{2}$ is $<0 - 0, -2 - (-3), 3 - 0> = <0, 1, 3>$ and vector $v_{2}$ is $<0 - (-3), -3 - (-3), 0 - 0> = <3, 0, 0>$.\vspace{0.25cm}

\par\noindent\large Using this information, we can find the normal vectors for $P_{1}$ and $P_{2}$, named $n_{1}$ and $n_{2}$ respectively, using the cross product of the $u$ and $v$ vectors of each plane. \vspace{0.25cm}

\par\noindent\large \begin{center}$n_{1} = u_{1} \times v_{1} = \begin{vmatrix}
\textbf{i} & \textbf{j} & \textbf{k} \\ 
1 & 0 & 3 \\ 
0 & -3 & 0  \notag
\end{vmatrix} = -9\textbf{i} - 3\textbf{k}$, $n_{2} = u_{2} \times v_{2} = \begin{vmatrix}
\textbf{i} & \textbf{j} & \textbf{k} \\ 
0 & 1 & 3 \\ 
3 & 0 & 0  \notag
\end{vmatrix} = 9\textbf{j} - 3\textbf{k}$ \end{center}\vspace{0.25cm}

\par\noindent\large From the above we get $n_{1} = <-9, 0, -3>$ and $n_{2} = <0, 9, -3>$.  Since we have two normal vectors, we can find the angle $\theta$ between the two planes by simply finding the angle between $n_{1}$ and $n_{2}$ using:
\begin{center}
    \Large $cos(\theta) = \frac{\abs{n_{1} \cdot n_{2}}}{\norm{n_{1}} \norm{n_{2}}}$
\end{center}

\par\noindent\large $\abs{n_{1} \cdot n_{2}} = (-9)(0) + (0)(9) + (-3)(-3) = 9$, $\norm{n_{1}} = \sqrt{(-9)^{2} + (0)^{2} + (-3)^{2}} = \sqrt{81 + 9} = \sqrt{90} = 3\sqrt{10}$, and $\norm{n_{2}} = \sqrt{(0)^{2} + (9)^{2} + (-3)^{2}} = \sqrt{81 + 9} = \sqrt{90} = 3\sqrt{10}$ \vspace{0.25cm}

\par\noindent\large Plugging in, we get \Large $cos(\theta) = \frac{9}{3\sqrt{10} 3\sqrt{10}} = \frac{9}{90}$.  \large Therefore, by taking the inverse cosine of this equation, we get \Large $\theta = cos^{-1}(\frac{9}{90}) = 84.26$ degrees.\vspace{0.25cm}

\begin{center}
\textbf{A drawn diagram is attached on the next page.}
\end{center}
\end{document}