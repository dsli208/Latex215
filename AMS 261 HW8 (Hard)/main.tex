\documentclass{article}
\usepackage[utf8]{inputenc}
\usepackage{geometry}
\usepackage{amsmath}
\usepackage{physics}
\geometry{legalpaper, portrait, margin = 0.5in}
\rmfamily

\title{AMS 261 HW8 (Hard)}
\author{David S. Li (SBUID: 110328771)}
\date{October 17, 2018}

\begin{document}

\maketitle

\section{14.1.70 - Use each order of integration to write an iterated integral that represents the area of the region $R$}
\par\noindent\large The region that we want to minimize is essentially the difference between the two functions in the given graph: $g(x) = \sqrt{x}$ and $f(x) = \frac{x}{2}$.

\subsection{Area = $\int\int dxdy$}
\par\noindent\large First, we need to identify our bounds.  We are integrating over $y$ in the outer-most integral, so our interval for that would be $0 \leq y \leq 2$.  Now we will need to integrate over $x$ in terms of $y$ for the inmost integral.  To do that, we use the two equations that bound the region and rearrange them in terms of $x$.  We get $x = 2y$ as our lower bound and $x = y^{2}$ as our upper bound.\vspace{0.25cm}

\par\noindent\Large Our integral becomes $\int_{0}^{2}\int_{2y}^{y^{2}}dxdy$
\subsection{Area = $\int\int dydx$}

\par\noindent\large Using the same approach as the previous part, first identify the bounds of $x$: $0 \leq x \leq 4$, then identify our bounds of $y$ in terms of $x$: $\frac{x}{2} \leq y \leq \sqrt{x}$.\vspace{0.25cm}

\par\noindent\Large Our integral is $\int_{0}^{4}\int_{\frac{x}{2}}^{\sqrt{x}} dydx$

\section{14.2.60 - Let the plane region $R$ be a unit circle and let the maximum value of $f$ on $R$ be 6.  Is the greatest possible value of $\int_{R}\int f(x, y) dydx$ equal to 6?  Why or why not?  If not, what is the greatest possible value?}

\par\noindent\large The greatest possible value cannot be 6; it was stated that the maximum value of $f(x, y)$ at any of possibly infinitely many $(x, y)$ pairs is 6; since $\int_{R}\int f(x, y) dydx$ is \textbf{the sum} of all values of $f$ over the entirety of $R$, then the value of that has to be \textbf{greater} than 6.  To get the greatest possible value, we need to set up the inequality $f(x, y) > 0$ and get bounds of $x$ and $y$ that produce a positive value for $f(x, y)$ and integrate $f(x, y)$ over those bounds.

\section{14.2.72 - Determine the region $R$ in the $xy$-plane that minimizes the value of the integral $\int_{R}\int (x^{2} + y^{2} - 4) dA$}

\par\noindent\large We want to get the set of values that make up $R$ such that our integrand $x^{2} + y^{2} - 4$ is \textbf{less than} 0.
\par\noindent So set up the inequality $x^{2} + y^{2} - 4 < 0$.  \textbf{Therefore, we get region $R$ as $x^{2} + y^{2} < 4$}\vspace{0.25cm}

\par\noindent (If we were actually to integrate, we need to figure out what to integrate over. Using $dydx$, we can say that the value of $x$ is in the range of $0 \leq x \leq 4$, and that $-\sqrt{4 - x^{2}} \leq y \leq \sqrt{4 - x^{2}}$.)\vspace{0.25cm}

%\par\noindent\Large So we have $\int_{0}^{4}\int_{-\sqrt{4 - x^{2}}}^{\sqrt{4 - x^{2}}}(x^{2} + y^{2} - 4) dydx$ as our integral to evaluate to get $R$.

%\par\noindent\large The optimal region for \textbf{minimizing} $R$ is getting something that is inside the circle that would be generated by the above integral, hence our inequality $x^{2} + y^{2} - 4 < 0$.)
\end{document}
